\documentclass[12pt,letterpaper]{article}

% Preamble for co456_notes.tex

\usepackage[dvipsnames, table]{xcolor}

\usepackage{amsmath}
\usepackage{amssymb}
\usepackage{amsthm}
\usepackage{changepage}
\usepackage{enumitem}
\usepackage{fancyhdr}
\usepackage{forest}
\usepackage{fullpage}
\usepackage{geometry}
\usepackage{mathrsfs}
\usepackage{mathtools}
\usepackage{parskip}
\usepackage[notmath]{sansmathfonts}
\usepackage{tabularx}
\usepackage[most]{tcolorbox}
\usepackage{tikz}
\usepackage{titlesec}
\usepackage{titletoc}
\usepackage[titles]{tocloft}

\usepackage[
  pdftitle={CO 456 Notes},
  pdfsubject={University of Waterloo, Fall 2020 (Martin Pei)},
  pdfauthor={Marco Yang <kc4yang@uwaterloo.ca>},
  colorlinks=true,
  linkcolor=blue
]{hyperref}
\usepackage[nameinlink]{cleveref}

\usetikzlibrary{arrows.meta}

%% Layout

% Margins
\geometry{
  margin=1in,
  headheight=1ex + \baselineskip,
  headsep=\baselineskip
}

% Header/footer
\pagestyle{fancy}
\fancyhf{}
\renewcommand{\headrulewidth}{0pt}
\renewcommand{\sectionmark}[1]{\markboth{\thesection\hspace{1.5ex}{#1}}{}}
\fancyhead[L]{\color{black!50} \small \sffamily CO 456}
\fancyhead[R]{\color{black!50} \small \sffamily \leftmark}
\fancyfoot[C]{\color{black!50} \small \sffamily \thepage}

% Sections (lectures)
\newcommand{\sectionbreak}{\clearpage\phantomsection}
\titleformat{\section}
  {\Large \sffamily \bfseries} % Format
  {Lecture \thesection:} % Label
  {1ex} % Sep
  {} % Before
\cftsetindents{section}{0pt}{2em}

% Subsections (topics)
\titleformat{\subsection}
  {\large \sffamily \bfseries} % Format
  {} % Label
  {0pt} % Sep
  {} % Before
\titlecontents{subsection}
  [2em] % Left spacing
  {} % Above
  {} % Numbered format
  {} % Numberless format
  {\hfill\contentspage} % Filler page format
  [] % Below

%% Commands

% Environments
\newtcbtheorem[number within=section]
  {thm} % environment name
  {Theorem} % display name
  { % options
    colback=blue!10,
    colframe=blue!10,
    colbacktitle=blue!10,
    coltitle=blue!60!black,
    fonttitle=\sffamily\bfseries,
    sharp corners,
    boxsep=1ex,
    toptitle=1ex,
    before skip=\baselineskip,
    after skip=\baselineskip,
    separator sign={~---},
    label type=thm
  }
  {thm} % label prefix
\crefname{thm}{Theorem}{Theorems}

\newtcbtheorem[use counter from=thm]
  {lem} % environment name
  {Lemma} % display name
  { % options
    colback=blue!10,
    colframe=blue!10,
    colbacktitle=blue!10,
    coltitle=blue!60!black,
    fonttitle=\sffamily\bfseries,
    sharp corners,
    boxsep=1ex,
    toptitle=1ex,
    before skip=\baselineskip,
    after skip=\baselineskip,
    separator sign={~---},
    label type=lem
  }
  {lem} % label prefix
\crefname{lem}{Lemma}{Lemmas}

\newtcbtheorem[use counter from=thm]
  {cor} % environment name
  {Corollary} % display name
  { % options
    colback=blue!10,
    colframe=blue!10,
    colbacktitle=blue!10,
    coltitle=blue!60!black,
    fonttitle=\sffamily\bfseries,
    sharp corners,
    boxsep=1ex,
    toptitle=1ex,
    before skip=\baselineskip,
    after skip=\baselineskip,
    separator sign={~---},
    label type={cor}
  }
  {cor} % label prefix
\crefname{cor}{Corollary}{Corollaries}

\newtcbtheorem[use counter from=thm]
  {exer} % environment name
  {Exercise} % display name
  { % options
    colback=red!10,
    colframe=red!10,
    colbacktitle=red!10,
    coltitle=red!60!black,
    fonttitle=\sffamily\bfseries,
    sharp corners,
    boxsep=1ex,
    toptitle=1ex,
    before skip=\baselineskip,
    after skip=\baselineskip,
    separator sign={~---},
    label type={exer}
  }
  {exer} % label prefix
\crefname{exer}{Exercise}{Exercises}

\newtcbtheorem[no counter]
  {defn} % environment name
  {Definition} % display name
  { % options
    nameref/.style={},
    colback=green!10,
    colframe=green!10,
    colbacktitle=green!10,
    coltitle=green!60!black,
    fonttitle=\sffamily\bfseries,
    sharp corners,
    boxsep=1ex,
    toptitle=1ex,
    before skip=\baselineskip,
    after skip=\baselineskip,
    separator sign={~---},
    label type={defn}
  }
  {defn} % label prefix
\crefname{defn}{Definition}{Definitions}

\newtcolorbox{ex}{
  enhanced,
  parbox=false,
  sharp corners,
  boxrule=0pt,
  left=1ex + 2mm + 4pt,
  right=0pt,
  bottom=0pt,
  frame hidden,
  title=Example,
  fonttitle=\sffamily\bfseries,
  colback=white,
  coltitle=red!60!black,
  colbacktitle=white,
  borderline west={4pt}{0pt}{red!60!black},
  before skip=\baselineskip,
  after skip=\baselineskip
}

\makeatletter
\newenvironment{proofb}{%
  \par
  \pushQED{\qed}
  \normalfont \topsep0\p@\@plus6\p@\relax
  \trivlist
  \item[]\ignorespaces
}{%
  \popQED\endtrivlist\@endpefalse
}
\makeatother

\newenvironment{thmproof}[1][Proof.]{
  \begin{tcolorbox}[
    enhanced,
    breakable,
    parbox=false,
    sharp corners,
    boxrule=0pt,
    left=1ex + 2mm + 4pt,
    right=0pt,
    bottom=0pt,
    frame hidden,
    title={#1},
    fonttitle=\sffamily\itshape,
    colback=white,
    coltitle=blue!60!black,
    colbacktitle=white,
    borderline west={4pt}{0pt}{blue!60!black},
    before skip=\baselineskip,
    after skip=\baselineskip
  ]
  \begin{proofb}
}{
  \end{proofb}
  \end{tcolorbox}
}

\newenvironment{exerproof}[1][Proof.]{
  \begin{tcolorbox}[
    enhanced,
    breakable,
    parbox=false,
    sharp corners,
    boxrule=0pt,
    left=1ex + 2mm + 4pt,
    right=0pt,
    bottom=0pt,
    frame hidden,
    title={#1},
    fonttitle=\sffamily\itshape,
    colback=white,
    coltitle=red!60!black,
    colbacktitle=white,
    borderline west={4pt}{0pt}{red!60!black},
    before skip=\baselineskip,
    after skip=\baselineskip
  ]
  \begin{proofb}
}{
  \end{proofb}
  \end{tcolorbox}
}

% Emphasis
\newcommand{\hldef}[1]{\textcolor{green!60!black}{\textbf{#1}}}

% Circled numbers
\newcommand*\circled[1]{
  \tikz[baseline=(char.base)]{
    \node[shape=circle, draw, inner sep=2pt] (char) {\footnotesize #1};
  }
}

% Enum with circled numbers
\newenvironment{enumcase}{
  \begin{enumerate}[label=\protect\circled{\arabic*}]
}{
  \end{enumerate}
}

%% Math commands

% Useful delimiters: abs, ceil, floor
\DeclarePairedDelimiter\abs{\lvert}{\rvert}
\DeclarePairedDelimiter\norm{\lVert}{\rVert}
\DeclarePairedDelimiter\ceil{\lceil}{\rceil}
\DeclarePairedDelimiter\floor{\lfloor}{\rfloor}

% Operators and such
\newcommand{\xor}{\oplus}
\DeclareMathOperator*{\mex}{mex}


%---------------
\begin{document}
%---------------

%%%%% Title
\title{
  \textbf{CO 456: Introduction to Game Theory} \\[\baselineskip]
  \large
  University of Waterloo \\
  Martin Pei \\
  Fall 2020
}
\author{Marco Yang}
\date{Last updated: \today}

{
  \sffamily
  \maketitle
}
\thispagestyle{empty}

%%%%% Table of contents
\pagebreak
\pagenumbering{roman}
\setcounter{page}{2}

{
  \sffamily
  \tableofcontents{\markboth{\contentsname}{}}
}

%%%%% List of theorems


%%%%% Lectures
\pagebreak
\pagenumbering{arabic}

%%%%% Lec 1
\section{Course Administration}

See the course outline.

%%%%% Lec 2
\section{Impartial Games}

\subsection{Nim}

We are given some piles of chips.
Two players play alternately.
On a player's turn, they pick a pile and remove at least 1 chip from it.
The first player who cannot make a move loses.

\begin{ex}
  \begin{itemize}
    \item
      1, 1, 2.
      \begin{itemize}
        \item Player I removes 2 chips from the last pile.
        \item Player II removes a 1-chip pile.
        \item Player I removes the last chip.
        \item Player II has no move and loses.
        \item Player I has a winning strategy.
      \end{itemize}
      This is a \hldef{winning game} or \hldef{winning position}.
    \item
      5, 5.
      \begin{itemize}
        \item Regardless of Player I's move, Player II can mirror it on the other pile.
        \item Player II always has a move, so Player I loses.
        \item Player I always loses (\emph{i.e.} Player II has a winning strategy).
      \end{itemize}
      This is a \hldef{losing game} or \hldef{losing position}.
    \item
      5, 7.
      \begin{itemize}
        \item Player I first equalizes piles (here, removing 2 from the pile of 7).
        \item Player II loses (by the previous case).
      \end{itemize}
      This is a winning game.
  \end{itemize}
\end{ex}

\begin{lem}{}{2.1}
  In instances of Nim with two piles of $n, m$ chips, it is a winning game if and only if
  $n \neq m$.
\end{lem}

\pagebreak
\subsection{Impartial games}

Nim is an impartial game.

\begin{defn}{impartial game}{impartial_game}
  Conditions for an \hldef{impartial game}:
  \begin{enumerate}
    \item There are two players, Player I (who starts) and Player II.
    \item There are several positions and a starting position.
    \item A player performs one of a set of allowable moves, which depends only on the current
      position and not on the player (hence ``impartial'').
      Each possible move generates an \hldef{option}.
    \item The players move alternately.
    \item There is complete information.
    \item There are no chance moves.
    \item The first player with no available move loses.
    \item The rules guarantee that games end.
  \end{enumerate}
\end{defn}

\begin{ex}
  Games which are not impartial:
  \begin{itemize}
    \item Tic-tac-toe (violates 7---may tie)
    \item Chess (violates 3---may only move your pieces)
    \item Poker (violates 5---cards hidden)
    \item Monopoly (violates 6---relies on dice rolls)
  \end{itemize}
\end{ex}

\begin{ex}
  Let $G = (1, 1, 2)$ be a Nim game.
  There are 4 possible moves, hence 4 possible options.

  \begin{center}
    \begin{forest}
      for tree={draw, rectangle, parent anchor=east, child anchor=west, edge={->}},
      l sep=2cm
      [{$G = (1, 1, 2)$}, grow'=east
        [{$H_1 = (1, 1, 1)$}]
        [{$H_2 = (1, 1, 0)$}]
        [{$H_3 = (1, 0, 2)$}]
        [{$H_4 = (0, 1, 2)$}]
      ]
    \end{forest}
  \end{center}

  Each $H_i$ is itself another Nim game.
\end{ex}

Note: we can define an impartial game by its position and options recursively.

\begin{defn}{game simplicity}{game_simple}
  A game $H$ that is reachable from game $G$ by a sequence of allowable moves is \hldef{simpler}
  than $G$.
\end{defn}

\begin{ex}
  Other impartial games:
  \begin{itemize}
    \item
      Subtraction game
      \begin{itemize}
        \item One pile of chips.
        \item Valid move: remove 1, 2, or 3 chips.
      \end{itemize}
    \item
      Rook game
      \begin{itemize}
        \item $m \times n$ chess board with a rook at $(i, j)$.
        \item Valid move: move the rook any number of spaces up or left.
      \end{itemize}
    \item
      Green hackenbush game
      \begin{itemize}
        \item Graph connected to the floor at some vertices.
        \item Valid move: remove an edge of the graph, then any components no longer connected to
          the floor.
      \end{itemize}
  \end{itemize}
\end{ex}

Spoiler: all impartial games are essentially Nim games.

\pagebreak
\subsection{Winning strategy}

\begin{lem}{}{2.2}
  In any game $G$, either Player I or Player II has a winning strategy.
\end{lem}

\begin{thmproof}
  By induction on simplicity of $G$.

  If $G$ has no allowable moves, then Player I loses, so Player II has a winning strategy.
  Assume $G$ has allowable moves and the lemma holds for all games simpler than $G$.
  Among all options of $G$, if Player I has a winning strategy in one of them, Player I will move to
  that option and win.
  Otherwise, Player II has a winning strategy for all options, so Player II wins regardless of
  Player I's move.
\end{thmproof}

That is, every impartial game $G$ is either a winning game or a losing game.

\begin{ex}
  Winning game of Nim (at least one winning move):
  \begin{center}
    \begin{forest}
      l sep=30pt,
      s sep=20pt
      [{$(5, 7)$}
        [{$(5, 5)$}, edge={->},
          edge label={node[midway, left, xshift=-1em, font=\footnotesize]{winning move}}]
        [{$(5, 6)$}, edge={->, dashed}]
        [{$(5, 7)$}, edge={->, dashed}]
        [{$\phantom{(}\cdots$\phantom{)}}, edge={->, dashed},
          edge label={node[midway, right, xshift=1em, font=\footnotesize]{non-winning moves}}]
      ]
    \end{forest}
  \end{center}

  Losing game of Nim (no winning moves):
  \begin{center}
    \begin{forest}
      l sep=30pt,
      s sep=20pt
      [{$(5, 5)$}
        [{$(4, 5)$}, edge={->, dashed}]
        [{$(3, 5)$}, edge={->, dashed}]
        [{$\phantom{(}\cdots$\phantom{)}}, edge={->, dashed}]
        [{$(5, 4)$}, edge={->, dashed}]
        [{$(5, 3)$}, edge={->, dashed}]
        [{$\phantom{(}\cdots$\phantom{)}}, edge={->, dashed}, edge label={
          node[midway, right, xshift=2em, font=\footnotesize]{non-winning moves}}]
      ]
    \end{forest}
  \end{center}
\end{ex}

Note: we assume players play perfectly.
If there is a winning move, then they will take it.

%%%%% Lec 3
\section{Equivalent Games (I)}

\subsection{Game sums}

\begin{defn}{game sum}{game_sum}
  Let $G$ and $H$ be two games with respective options $G_1, \ldots, G_m$ and $H_1, \ldots, H_n$.
  We define \hldef{$G + H$} as the game with options
  \[
    G_1 + H, \ldots, G_m + H, G + H_1, \ldots, G + H_n.
  \]
\end{defn}

\begin{ex}
  We denote \hldef{$*n$} to be a game of Nim with one pile of $n$ chips.
  Then $*1 + *1 + *2$ is a Nim game with three piles of 1, 1, and 2 chips.
\end{ex}

\begin{ex}
  Let \hldef{$\#n$} be the subtraction game with $n$ chips.
  Then $*5 + \#7$ is the game where a move is either to remove at least 1 chip from the pile of 5
  (Nim) or to remove 1, 2, or 3 chips from the pile of 7 (subtraction game).
\end{ex}

\begin{lem}{}{3.1}
  Let $\mathcal{G}$ be the set of all impartial games.
  Then for all $G, H, J \in \mathcal{G}$,
  \begin{enumerate}
    \item $G + H \in \mathcal{G}$ (closure)
    \item $(G + H) + J = G + (H + J)$ (associativity)
    \item There exists an identity $0 \in \mathcal{G}$ (the game with no options) where
      $G + 0 = 0 + G = G$.
    \item $G + H = H + G$ (symmetry)
  \end{enumerate}
\end{lem}

Note: $\mathcal{G}$ is an abelian group, except for an inverse element.

\pagebreak
\subsection{Game equivalences}

\begin{defn}{game equivalence}{game_equiv}
  Two games $G$ and $H$ are \hldef{equivalent} if for any game $J$, $G + J$ and $H + J$ have the
  same outcome (\emph{i.e.}, both are winning games or both are losing games).
  Notation: $G \equiv H$.
\end{defn}

\begin{ex}
  $*3 \equiv *3$.
  Since $*3 + J$ is the same game as $*3 + J$, they have the same outcome.

  $*3 \not\equiv *4$.
  $*3 + *3$ is a winning game but $*4 + *3$ is a losing game (\cref{lem:2.1}).
\end{ex}

\begin{lem}{}{3.2}
  $*n \equiv *m$ if and only if $n = m$.
\end{lem}

\begin{lem}{}{3.3}
  The relation $\equiv$ is an equivalence relation.
  That is, for all $G, H, K \in \mathcal{G}$,
  \begin{enumerate}
    \item $G \equiv G$ (reflexivity).
    \item $G \equiv H$ if and only if $H \equiv G$ (symmetry).
    \item If $G \equiv H$ and $H \equiv J$, then $G \equiv J$ (transitivity).
  \end{enumerate}
\end{lem}

\begin{exer}{}{3.4}
  Prove that if $G \equiv H$, then $G + J \equiv H + J$ for any game $J$.
\end{exer}

\begin{exerproof}
  Consider any games $J$ and $K$.
  Let $M = J + K$.
  Then $G + J + K = G + M$ and $H + J + K = H + M$.
  Since $G \equiv H$, $G + M$ and $H + M$ have the same outcome.
  But then $(G + J) + K$ and $(H + J) + K$ have the same outcome by \cref{lem:3.1}.
  $K$ was arbitrary, so $G + J \equiv H + J$.
\end{exerproof}

\pagebreak
\subsection{Losing games and empty Nim}

Nim with one pile $*n$ is a losing game if and only if $n = 0$.

\begin{thm}{}{3.5}
  $G$ is a losing game if and only if $G \equiv *0$.
\end{thm}

\begin{cor}{}{3.6}
  If $G$ is a losing game, then $J$ and $J + G$ have the same outcome for any game $J$.
\end{cor}

\begin{thmproof}
  Since $G$ is a losing game, $G \equiv *0$ by \cref{thm:3.5}.
  Then $J + G \equiv J + *0 \equiv J$ by \cref{exer:3.4} and \cref{lem:3.1}, so $J$ and $J + G$ have
  the same outcome.
\end{thmproof}

\begin{ex}
  \begin{enumerate}
    \item
      Recall $*5 + *5$ and $*7 + *7$ are losing games.
      \cref{cor:3.6} says $*5 + *5 + *7 + *7$ is also a losing game.
      (Player I moves in either $*5 + *5$ or $*7 + *7$.
      Player II plays a winning move in the same part by equalizing piles.)
    \item
      $\underbrace{*1 + *1 + *2}_{\text{winning}} + \underbrace{*5 + *5}_{\text{losing}}$ is a
      winning game by \cref{cor:3.6}.
      (Player I removes $*2$, leaving a similar game to the previous.)
  \end{enumerate}
\end{ex}

\begin{thmproof}[Proof (\cref{thm:3.5}).]
  \begin{itemize}[leftmargin=5em]
    \item[$(\impliedby)$]
      If $G \equiv *0$, then $G + *0$ has the same outcome as $*0 + *0$.
      But $*0 + *0$ is a losing game, so $G$ is a losing game.
    \item[$(\implies)$]
      Suppose $G$ is a losing game.
      We show $G + J$ and $*0 + J \equiv J$ have the same outcome.
      \begin{enumcase}
        \item
          Suppose $J$ is a losing game.
          We show ``if $G$ and $J$ are both losing games, then $G + J$ is a losing game'' by
          induction on simplicity of $G + J$.

          When $G + J$ has no options, $G$ and $J$ have no options, so $G$, $J$, and $G + J$ are all
          losing games.
          Assume $G + J$ has some otpions and the statement holds for simpler games.
          WLOG, Player I moves on $G$, resulting in $G' + J$.
          $G$ is a losing game, so $G'$ is a winning game.
          Player II makes a winning move from $G'$ to $G''$, resulting in $G'' + J$.
          Then $G''$ is a losing game, so by induction $G'' + J$ is a losing game.
          Player I loses, so $G + J$ is a losing game.
        \item
          Suppose $J$ is a winning game, so $J$ as a winning move to $J'$.
          Player I moves from $G + J$ to $G + J'$.
          $G$ and $J'$ are both losing games, so by \circled{1} $G + J'$ is a losing game.
          Player II loses, so Player I wins and $G + J$ is a winning game.
      \end{enumcase}
  \end{itemize}
\end{thmproof}

%---------------
\end{document}
%---------------
